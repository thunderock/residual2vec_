%! Author = ashutosh
%! Date = 8/17/22

\documentclass{article}

\usepackage[english]{babel}
\usepackage[utf8]{inputenc}
\usepackage[letterpaper]{geometry}
\usepackage{amsmath, amssymb, enumerate, physics}
\usepackage[colorlinks=true, allcolors=blue]{hyperref}
\usepackage{epstopdf}
\usepackage{xcolor}
\usepackage{enumitem}
\usepackage{textcomp}
\usepackage[T1]{fontenc}
\usepackage[utf8]{inputenc}
% Standard package includes
\usepackage{times}
\usepackage{latexsym}
\hypersetup{
    colorlinks=true,
    linkcolor=blue,
    filecolor=magenta,
    urlcolor=cyan,
    }

\title{\mathbf{Independent Study Project Proposal}}
\author{{Ashutosh Tiwari (ashutiwa@iu.edu)}}

\begin{document}
\maketitle

\section{Project}

\subsection{Debiased Walk: Learning Representations using Debiased Embeddings} \label{project_name}

\section{Number of work hours}
12 hours / week (equivalent to 3 credit hours)

\section{Required meetings}
2-3  meetings / week

\section{Required Readings and Assignments with due dates}
\subsection{Readings} ~\label{readings}
\begin{itemize}
    \item ~\cite{khajehnejad_crosswalk_2021} ($1^{st}$ week)
    \item ~\cite{rahman_fairwalk_2019} ($2^{nd}$ week)
    \item ~\cite{laclau_all_2021} ($3^{rd}$ week)
    \item ~\cite{gonen_lipstick_2019}($4^{th}$ week)
    \item ~\cite{bolukbasi_man_2016}($5^{th}$ week)
    \item ~\cite{DBLP:journals/corr/PerozziAS14}($6^{th}$ week)
    \item ~\cite{ravfogel_null_2020}($7^{th}$ week)
    \item ~\cite{garg_word_2018}($8^{th}$ week)
    \item ~\cite{kojaku_residual2vec_2021}($9^{th}$ week)
    \item ~\cite{brunet_understanding_2019}($10^{th}$ week)
    \item ~\cite{kenna_using_2021}($11^{th}$ week)
    
\end{itemize}

\subsection{Assignments}
\begin{itemize}
\item Investigate and possibly publish work on comparison of different methods to debias graph embeddings with residual2vec. Possibly write utilities which facilitate these experiments by allowing plug and play of models, techniques, datasets and configurations. (first half)
\item Investigate and possibly publish work on a framework to analyze the bias manifold structure of different kinds of biases in different datasets using graph embeddings generated by different models.(second half)
\end{itemize}

\section{Assessment}
Student is going to be assessed on the amount of understanding he gains and the development work he undertakes during this study. This study banks upon the ability of student to be curious about a rather fundamental and at the same time under investigated topic in machine learning.

\section{Work Plan}

Primary purpose of this study is to understand and contribute to the understanding the manifold of existing classes of biases. Historically it is assumed that most of those are linear in nature and hence most of metrics used to quantify those are linear in nature. But there is sufficient evidence to doubt that size of embeddings, type of dataset do play a significant role in determining same.

Second part of this study explores different methods of tackling these biases. There already exist different methods (some included in reading list~\ref{readings}) some of which work on embeddings while some on data itself. Study also focuses on comparing all these methods. In particular we are interested use of residual2vec as such a method.


\bibliographystyle{apalike}

\bibliography{ashutiwa}
\end{document}